\pagebreak
\section{Preface}

\paragraph{Welcome to the ECE 100 - Intro to Profession - Laboratory Section}{
In this laboratory, students will utilize Arduinos, sensors, code, and other items to build an assortment of robots. Students will collaborate in groups and participate in a variety of competetions between groups from all lab sections}

\paragraph{Academic Honesty}{
All students must follow the IIT Code of Academic Honesty during this laboratory session. Full text for the Code of Academic Honesty can be found on \href{https://www.iit.edu/student-affairs/student-handbook/fine-print/code-academic-honesty}{IIT's website}}

\paragraph{Reference Material}{
There are several good reference sites for coding with Arduino and Arduino Sensors. The Arduino IDE uses the C++ coding language, so any reference material for C or C++ is also applicable
\begin{itemize}
	\item \href{https://www.arduino.cc/reference/en/}{Arduino Official Language Reference}
	\item \href{https://www.arduino.cc/en/tutorial/sketch}{Arduino Tutorial Sketch}
	\item \href{https://www.tutorialspoint.com/arduino/index.htm}{Tutorialspoint.com Arduino Tutorials}
\end{itemize}
}

\paragraph{Reports Format}{ Templates for Prelab and Postlab reports can be found on Blackboard. Reports may be written using \href{https://www.latex-project.org/}{LaTeX} or a text processor such as Microsoft Word, Google Docs, etc.
}